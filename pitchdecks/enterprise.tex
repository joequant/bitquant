\documentclass{beamer}
\title {Blockchain for the Enterprise}
\author{Joseph Chen-Yu~Wang}
\institute{Bitquant Research Laboratories (Asia) Limited\\
  in association with the Banzai Institute of Advanced Research\\
  and the Hong Kong Caviliers}
\date{\today}
\usetheme{Dresden}
\usecolortheme{beaver}
\setlength{\parskip}{\baselineskip}
\begin{document}
\frame{\titlepage}
\begin{frame}
  \frametitle{What the enterprise needs}
  \begin{itemize}
  \item Protocols to do private transactions on public blockchains
  \item Office interfaces (the ability to handle documents,
    spreadsheets, and databases)
  \item Scalable identity and authentication model
  \item Convince your workers this won't make their lives worse
  \end{itemize}
\end{frame}
\begin{frame}
  \frametitle{Private blockchains will not work}
  \begin{itemize}
  \item Private blockchains will not work
  \item It is impossible to maintain total trust between more than a
    small number of parties.  You must assume that an attacker will be
    able to control one node of a blockchain
  \item Example: You have a network of banks.  You must assume that an
    attacker can gain control over one bank's systems (because this
    has happened before)
  \item If you can design security protocols that are resistant to an
    attacker gaining control over one node, then there is no need to
    make the system private
  \item Desire for private blockchain based on maintaining an
    unsustainable business model
  \end{itemize}
\end{frame}
\begin{frame}
  \frametitle{The blockchain paradox}
  \begin{itemize}
    \item Private blockchains assume that the members of a consortia
      trust each other
    \item If they *did* trust each other, then they can just dump the
      data into a central database
    \item But the reason they haven't is that they don't trust each other
    \item Blockchain is most useful for linking data between entities
      that don't trust each other
    \item Once you figure out a way that people that don't trust each
      other can share data, then you can just open the system up to
      everyone else that you don't trust (i.e. the entire world)
  \end{itemize}
\end{frame}
\begin{frame}
  \frametitle{A private blockchain makes as much sense as a private internet}
  \begin{itemize}
    \item We conduct sensitive business transactions over public
      networks (i.e. cell phones, e-mail, web)
    \item We do not build our own internet to maintain security
    \item We use cryptography and social protocols to make sure that
      the bad people can't get access to the information
    \item We create virtual private networks out of public networks
    \item Same philosophy should be used for blockchain
  \end{itemize}
\end{frame}
\begin{frame}
  \frametitle{Office interfaces}
  \begin{itemize}
    \item Businesses work by processing fundamental artifacts
      (documents, spreadsheets, and databases)
    \item Microsoft has made ten of billions by designing tools to
      process these artifacts (i.e. Microsoft Office)
    \item The blockchain needs interfaces so that it looks like a
      document, a spreadsheet, and a database
    \item Once those interfaces are designed, then you can layer
      existing processes on top of the blockchain
  \end{itemize}
\end{frame}
\begin{frame}
  \frametitle{Identity/authentication models}
  \begin{itemize}
  \item We do not have a scalable model for identity/authentication
  \item The two models are 1) everything is public or 2) atomic models
    (i.e. google/facebook/microsoft authentication)
  \item Within companies we have internal models that depend on layers
    of permissions, and central authortization to grant or revoke
    those privileges
  \item Identity/authentication models are bureaucratic and
    hierarchical because the social systems that they exist in are
    bureaucratic and hierarchical
  \end{itemize}
\end{frame}
\begin{frame}
  \frametitle{Characteristics of an identity/authentication model}
  \begin{itemize}
  \item Decentralized - No central authority
    \item Flexible - The things that you should be able to do or not
      to need to be user defined
    \item Non-hierarchical - Current identity models assume a
      hierarchical bureaucratic structure
    \item Private - You should be able to have authorization to do
      something without anyone knowing that you have permission to do
      something
    \item Non-intrusive - A good system will layer on top of existing
      social systems and not replace them
  \end{itemize}
\end{frame}
\begin{frame}
  \frametitle{Other things}
  \begin{itemize}
  \item How do you shred data on the blockchain?
  \item Do not expect technology to fix political problems.
  \item How can you integrate identity and authorization with human
    social mechanism (i.e. you hide identity behind a BVI
    corporation).
  \item The capital spend paradox - The companies that would most
    benefit from blockchain don't have the capital budget to develop
    it.  The companies with the capital budget to develop it, don't
    really have use cases that will require it.
  \item The employment paradox.  The people that you need to implement
    the system are the people that will be fired if it succeeds.  You
    need to sell that the system will not make their life worse.  This
    can involve some messy conversations.
  \end{itemize}
\end{frame}
\begin{frame}
  \frametitle{Recommendations for Western banks}
  \begin{itemize}
    \item Be brutally honest.  If this blockchain thing works, then a
      lot of people will lose their jobs.
    \item But those jobs are going to be lost eventually
    \item Set up blockchain systems so that people can use them to
      create jobs once they get fired
    \item Suicide squads - You are going to be fired in one year.
      Here is some cash so that you can work on a blockchain solution
      that will be of some benefit to the bank and yourself once you
      lose your job
  \end{itemize}
\end{frame}
\begin{frame}
  \frametitle{Hong Kong will be the center of blockchain}
  \begin{itemize}
    \item The fundamental use case for blockchain is to allow entities
      that don't trust each other to interoperate
    \item Asia consists of about twenty different entities that just
      don't trust each other
    \item Asia is growing with new business
    \item The world consists of governments, companies, and
      individuals that don't trust each other but have to work with
      each other nevertheless
    \item Hong Kong is one of the few places in the world that is
      neutral ground
  \end{itemize}
\end{frame}
\end{document}
