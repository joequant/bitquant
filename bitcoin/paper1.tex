\title{Hong Kong Reguation of Bitcoin}
\author{Joseph C Wang\\
\small Bitquant Research Laboratories
}
\begin{document}
\begin{abstract}
In recent months, bitcoin has received a lot of attention in the
media, and there has been much discussion online concerning the
regulatory aspects of bitcoin.  In this article, we survey the role of
bitcoin within the financial regulatory structure of Hong Kong playing
special attention to Hong Kong's unique position as an international
financial center within the People's Republic of China.  We examine
the relevant laws and regulation in Hong Kong concerning bitcoin and
the implications of bitcoin's designation as a virtual commodity.  We
then example the purpose of financial regulation and argue that the
current regulatory structure is well suited to those goals.  We then
examine the possible future of the regulation of bitcoin and list a
number of areas for attention by regulators and provide
recommendations for bitcoins possible future.
\end{abstract}

\section{Introduction}

Bitcoin is one of several cryptocurrencies.

\section{Hong Kong's Regulatory Environment}
Hong Kong has a unique position as a financial center which causes
financial regulation within Hong Kong to be different from those of
other jurisdiction.  Under the British rule, Hong Kong became a
financial trading center, and this position has been entrenched in
Hong Kong's mini-constitution, the Basic Law of Hong Kong which has
governed H

Specifically Article **** of the Basic Law states

\section{Regulation of Bitcoin in Hong Kong}



\subsection{Commodity Markets Prohibition Ordinance}


\subsection{Securities and Futures Ordinance}

The SFO lists a number of regulated activities

In addition to the regulated activities the SFO consid



\subsection{Exchange Fund Ordinance}

\subsection{Anti-Money Laundering and Counter-Terrorist Financing
(Financial Institutions) Ordinance Cap 615}

This ordinance which was passed in 2012 has two parts.  The first
section imposes record keeping requirements on all financial
institutions.  The second section creates a system for licensure for
money services, which are defined under Cap 615 s 24 and includes the
following definitions

currency - includes a cheque and a traveller's cheque

money means money in whatever form or currency

money changing service means a service for the exchaging of currencies
that is operaated in Hong Kong as a business 

money service means: a money changing service or a remittance service

The license system for a money services business is adminstered by the
Department of Customs and Excise, and the criterion for the issuance
of a license are those that related to identifying the ultimate
beneficial owners of the money service and whether the owners have
been convicted of money laundering, drug trafficking, terrorism or
other serious crime.  The criterion for license issuance are criminal
and character based and do not include the financial stability or
creditworthness of the business.

Based on the definitions in this ordinance, and the definition of
bitcoin as a commodity, no license appears to be necessary for a
business whose sole function is to convert bitcoin to and from Hong
Kong dollars, and the ordinance appears not to include gold dealers
and jewellers operating in Hong Kong.

However, a license may be necessary if the business intends to use
bitcoin as a mechanism to convert currencies within Hong Kong or to
transmit money outside of Hong Kong.

Once a company becomes a licensed money services operator, they are
then considered a financial institution, and must abide by the record
keeping and other provisions of the first section of the ordinance.

Although a money service license is not essential for operating a
bitcoin exchange, several of the exchanges in Hong Kong have applyed
for and received money services license, and have undertaking AML-KYC
examinations. 

\subsection{Deceptive Claims Ordinance}

The application of the deceptive claims ordinance to commodities can
be seen specifically in relation to the special provisions regarding
gold coin.

\subsection{Theft Ordinance}

\section{Goals of Regulation}

\subsection{International financial center}

The nature of the regulation regarding bitcoin is also consist with
the need to preserve Hong Kong's system of free enterprise, and a
business environment which is conducive to innovation.

\subsection{Macroeconomic stability}

\subsection{Investor protection}
There are currently no investor protections in Hong Kong concerning
bitcoin beyond those provided by general anti-fraud and anti-deception
statutes.  In order to examine whether this is an adequate regulatory
framework we must first examine why it is necessary to adopt special
rules regarding securties.

deterence is sufficient

fraud laws are focused on particular acts and not generalized harm

conflict of interest results in imprudence





\subsection{Anti money and anti-terrorism}

\section{Future Directions in Bitcoin Regulation}

\subsection{Anti-Money and Anti-Terrorism Laundering}

Because the rules are applicable to all financial institutions from
large to small, it appears unlikely that bitcoin will outgrow these
regulations regardless of the outcome.  If any changes to the AML
infrastructure is necessary, it will be an item that influences all
financial institution.  Specific regulations on bitcoin will only
prove necessary if bitcoin has some as yet unknown characteristic that
encourages money laundering and criminal activity is a manner that
regular currency does not.

\subsection{Macroeconomic stability}

The main concern in this area is that bitcoin will be part of a
shadow banking system in much the same way financial derivatives
created such as system before the 2008 financial crisis.

The effort to prevent this has consisted of separating the bitcoin
economy from the banking system.  

In determining whether further action is necessary it will be
sufficient to look at the total volume of bitcoin.  If a substantial
amount of economic activity becomes conducted in bitcoin, then further
legislation may be necessary.  In addition, further thinking as to the
regulatory structure of bitcoin may be necessary if financial
institutions desire entering the bitcoin business.

Additional gudiance from the HKMA may also be necessary as to the
types and quantity of business loans made to support bitcoin.


\end{document}
